\chapter{}

\topic{Maschke's theorem}

We now present another instance of the Jacobson semisimplicity problem.
In this case, our result is for finite groups. 

\begin{theorem}[Maschke]
\index{Maschke's theorem}
	Let $G$ be a finite group. Then $J(K[G])=\{0\}$ if and only 
	if the characteristic of $K$ is zero 
	or does not divide the order of $G$. 
\end{theorem}

\begin{proof}
	Assume that $G=\{g_1,\dots,g_n\}$, where $g_1=1$. Let 
	\[
	\rho\colon K[G]\to K,
	\quad
	\alpha\mapsto\trace(L_{\alpha}),
	\]
	where 
	$L_{\alpha}(\beta)=\alpha\beta$. Then 
	\[
	\rho(g_i)=\begin{cases}
	    n & \text{if $i=1$,}\\
	    0 & \text{if $2\leq i\leq n$},
	\end{cases}
	\]
	as $L_{g_i}(g_j)=g_{i}g_j\ne g_j$, the matrix of 
	$L_{g_i}$ in the basis $\{g_1,\dots,g_n\}$ contains zeros in the main diagonal. 

	Assume that $J=J(K[G])$ is non-zero and let 
	$\alpha=\sum_{i=1}^n\lambda_ig_i\in J\setminus\{0\}$. Without loss of generality
	we may assume that $\lambda_1\ne 0$ (if $\lambda_1=0$ there exists some 
	$\lambda_i\ne 0$ and we need to take $g_i^{-1}\alpha\in J$). Then 
	\[
		\rho(\alpha)=\sum_{i=1}^n \lambda_i\rho(g_i)=n\lambda_1.
	\]
	Since $G$ is finite, $K[G]$ is a finite-dimensional algebra and hence 
	$K[G]$ is left artinian. Since $J$ is a nilpotent ideal, 
	in particular, $\alpha$ is a nilpotent element. Then 
	$L_{\alpha}$ is nilpotent and hence $0=\rho(\alpha)=n\lambda_1$. This implies that
	the characteristic of the field $K$ divides $n$. 

	Conversely, let $K$ be a field of prime characteristic and that this prime divides 
	$n$. Let $\alpha=\sum_{i=1}^ng_i$. Since $\alpha
	g_j=g_j\alpha=\alpha$ for all $j\in\{1,\dots,n\}$, the set 
	$I=K[G]\alpha$ is an ideal of $K[G]$. Since, moreover,   
	\[
		\alpha^2=\sum_{i=1}^n g_i\alpha=n\alpha=0
	\]
	in the field $K$, it follows that $I$ is a nilpotent non-zero ideal. Thus $J(K[G])\ne\{0\}$, 
	as Proposition~\ref{pro:nilJ} yields $I\subseteq J(K[G])$.
\end{proof}

Since the Jacobson radical of a group algebra of a finite group contains 
every nil left ideal, the following consequence of the theorem follows immediately:

\begin{corollary}
	\label{cor:GfinitoNOnil}
	Let $G$ be a finite group. Then $K[G]$ does not contain non-zero nil left ideals. 
\end{corollary}


% \begin{proof}
% 	Es consecuencia inmediata del teorema de Maschke ya que $J(K[G])$ contiene a
% 	todo ideal a izquierda nil.	
% \end{proof}

%\index{Anillo!semisimple}
%Recordemos que un anillo unitario $R$ se dice \textbf{semisimple} si para cada
%ideal $I$ de $R$ existe un ideal $J$ de $R$ tal que $R=I\oplus J$.
%
%%\begin{corollary}
%%	Sea $G$ un grupo finito y $K$ un cuerpo de característica coprima con el
%%	orden de $G$. Entonces $K[G]$ es semisimple.
%%\end{corollary}
%%
%%\begin{proof}
%%	
%%\end{proof}
%
%\begin{theorem}
%	Si $G$ es un grupo infinito, entonces $K[G]$ nunca es semisimple.
%\end{theorem}
%
%\begin{proof}
%	Sea $R=K[G]$ y supongamos que $R$ es semisimple.  Si $I$ es el ideal de
%	aumentación de $R$, existe un ideal no nulo $J$ de $R$ tal que $R=I\oplus
%	J$. Como $R$ es unitario, existen $e\in I$, $f\in J$ tales que $1=e+f$. Si
%	$x\in I$, entonces $x=xe+xf$ y luego $xf=x-xe\in I\cap J=\{0\}$. Como
%	entonces $x=xe$ para todo $x\in I$, en particular $e_1=e_1^2$. Análogamente
%	vemos que $e_2^2=e_2$. Además $ef=0$ pues $ef\in I\cap J=\{0\}$.  Como $I$
%	es el ideal de aumentación y $If=(Re)f=R(ef)=0$, se concluye que $(g-1)f=0$
%	para todo $g\in G$ pues $g-1\in I$. Si suponemos que $f=\sum_{h\in
%	G}\lambda_hh$, entonces 
%	\[
%	f=gf=\sum_{h\in G}\lambda_h(gh)=\sum_{h\in
%	G}\lambda_{g^{-1}h}h.
%	\]
%	Luego $\lambda_h=\lambda_{g^{-1}h}$ para todo $g,h\in G$, una contradicción
%	pues como $f\ne 0$ la suma que define a $f$ es infinita. 
%\end{proof}


\topic{Herstein's theorem}

Our aim now is to answer the following question: When
a group algebra is algebraic? Herstein's theorem provides
a solution in the case of fields of characteristic zero. In prime characteristic,
the problem is still open. 

\begin{definition}
\index{Group!locally finite}
	A group $G$ is \textbf{locally finite} if every finitely generated 
	subgroup of $G$ is finite. 
\end{definition}

If $G$ is a locally finite group, then every element $g\in G$ has finite order, as
the subgroup $\langle g\rangle$ is finite because it is finitely generated.

\begin{example}
    Every finite group is locally finite
\end{example}

\begin{example}
    The group $\Z$ is not locally finite because it is torsion-free.
\end{example}

\begin{example}
\index{Pr\"ufer's group}
	Let $p$ be a prime number. 
	The \textbf{Pr\"ufer's group}  
	\[
		\Z(p^{\infty})=\{z\in\C:z^{p^n}=1\text{ for some $n\in\Z_{>0}$}\}, 
	\]
	is locally finite. 
\end{example}

\begin{example}
	Let $X$ be an infinite set and $\Sym_X$ be the set of bijective maps $X\to
	X$ moving only finitely many elements of $X$. Then 
	$\Sym_X$ is locally finite.
\end{example}

\index{Group!torsion}
A group $G$ is a \textbf{torsion} group if every element of $G$
has finite order. Locally finite groups are torsion groups. 

\begin{example}
    Abelian torsion groups are locally finite. Let $G$ be a locally finite abelian group 
    and $H$ be a finitely generated subgroup. Since $G$ is an abelian torsion group, so is $H$. Thus
    $H$ is finite by the structure theorem of abelian groups. 
\end{example}

\begin{proposition}
\label{pro:exact_LI}
	Let $G$ be a group and $N$ be a normal subgroup of $G$. If $N$ and $G/N$
	are locally finite, then $G$ is locally finite.
\end{proposition}

\begin{proof}
	Let $\pi\colon G\to G/N$ be the canonical map and $\{g_1,\dots,g_n\}$ be a finite subset of $G$. 
	Since $G/N$ is locally finite, the subgroup $Q$ of $G/N$ generated by 
	$\pi(g_1),\dots,\pi(g_n)$ is finite, say
	\[
		Q=\{\pi(g_1),\dots,\pi(g_n),\pi(g_{n+1}),\dots,\pi(g_m)\}
	\]
	for some $g_{n+1},\dots,g_m\in G$. 
	
	For each $i,j\in\{1,\dots,n\}$ there exist $u_{ij}\in N$ and 
	$k\in\{1,\dots,m\}$ such that \[
	g_ig_j=u_{ij}g_k.
	\]
	Let $U$ be the subgroup of $G$
	generated by $\{u_{ij}:1\leq i,j\leq n\}$. Since $N$ is locally finite, $U$ is finite. Moreover, since 
	each $g_ig_jg_l$ can be written as 
	\[
		g_ig_jg_l=u_{ij}g_kg_l=u_{ij}u_{kl}g_t=ug_t
	\]
	for some $u\in U$ and $t\in\{1,\dots,m\}$, it follows that the subgroup 
	$H$ of $G$ generated by $\{g_1,\dots,g_n\}$ is finite, as 
	$|H|\leq m|U|$. 
\end{proof}

\index{Group!solvable}
A group $G$ is
\textbf{solvable} if there exists a sequence
of subgroups 
\begin{equation}
	\label{eq:resoluble}
	\{1\}=G_0\subsetneq G_1\subsetneq \cdots\subsetneq G_n=G
\end{equation}
where each $G_i$ is normal in $G_{i+1}$ and each 
quotient $G_i/G_{i-1}$ is
abelian.

\begin{example}
    Abelian groups are solvable. 
\end{example}

Subgroups and quotients of solvable groups are solvable. 

\begin{example}
    Groups of order $<60$ are solvable.
\end{example}

\begin{example}
    $\Alt_5$ and $\Sym_5$ are not solvable. 
\end{example}

\index{Burnside's theorem}
\index{Feit--Thompson's theorem}
A famous theorem of Burnside states that 
groups of order $p^aq^b$ for prime numbers $p$ and $q$ are solvable 
A much harder theorem proved by Feit and Thompson states that
groups of odd order are solvable.

\begin{proposition}
	If $G$ is a solvable torsion group, 
	then $G$ is locally finite. 
\end{proposition}

\begin{proof}
	We proceed by induction on $n$, the length of the sequence~\eqref{eq:resoluble}. 
	If $n=1$, then $G$ is finite because it is abelian and a torsion group.
	Now assume the result holds for solvable groups of length $n-1$ and let
	$G$ be a solvable group with a sequence~\eqref{eq:resoluble}. Since $G_{n-1}$ is 
	a solvable torsion group, the inductive hypothesis implies that 
	$G_{n-1}$ is locally finite. Since $G/G_{n-1}$ is an abelian torsion group, 
	it is locally finite. The result now follows from Proposition \ref{pro:exact_LI}.
\end{proof}

We now prove Herstein's theorem.

\begin{theorem}[Herstein]
\index{Herstein's theorem}
	If $G$ is a locally finite group, then $K[G]$ is algebraic. Conversely, if 
	$K[G]$ is algebraic and $K$ has characteristic zero, then $G$ 
	is locally finite. 
\end{theorem}

\begin{proof}
	Assume that $G$ is locally finite. Let $\alpha\in K[G]$. The subgroup 
	$H=\langle\supp\alpha\rangle$ is finite, as it is finitely generated. Since 
	$\alpha\in K[H]$ and $\dim_KK[H]<\infty$, the set 
	$\{1,\alpha,\alpha^2,\dots\}$ is linearly dependent. Thus $\alpha$ is
	algebraic over $K$.

	Let $\{x_1,\dots,x_m\}$ be a finite subset of $G$. Adding inverses if needed,
	we may assume that $\{x_1,\dots,x_m\}$ generates the subgroup 
	$H=\langle x_1,\dots,x_m\rangle$ as a semigroup. Let 
	\[
	\alpha=x_1+\dots+x_m\in K[G].
	\]
	Since $\alpha$ is algebraic over $K$, 
	there exist $b_0,b_1,\dots,b_{n+1}\in K$ such that 
	\[
	b_0+b_1\alpha+\cdots+b_{n+1}\alpha^{n+1}=0,
	\]
	where $b_{n+1}\ne 0$. Rewrite this as 
	\[
		\alpha^{n+1}=a_0+a_1\alpha+\cdots+a_n\alpha^n
	\]
	for some $a_0,\dots,x_n\in K$. Let $w=x_{i_1}\cdots
	x_{i_{n+1}}\in H$ be a word of length $n+1$. 
	Note that 
	\[
		\alpha^{k}=(x_1+\cdots+x_m)^{k}
		=\sum x_{i_1}\cdots x_{i_{k}}
	\]
	for all $k$. 
	Two words $x_{i_1}\cdots x_{i_{k}}$ and 
	$x_{j_1}\cdots x_{j_{k}}$ could represent the same element of the group $H$. In this case, 
	the coefficient of $x_{i_1}\cdots x_{i_{k}}=x_{j_1}\cdots x_{j_{k}}$ 
	in $\alpha^{k}$ will be a positive integer $\geq2$.  
	
	Since $K$
	is of characteristic zero, it follows that $w\in\supp(\alpha^{n+1})$. Since, moreover,  
	$\alpha^{n+1}=\sum_{j=0}^na_j\alpha^j$, it follows that 
	$w\in\supp(\alpha^j)$ for some $j\in\{0,\dots,n\}$. Thus each
	word in the letters $x_j$ of length $n+1$ can be written as a word in the letters $x_j$ of 
	length $\leq n$. Therefore $H$ is finite and hence $G$ is locally finite. 
\end{proof}

\topic{Formanek's theorem, I}

\begin{exercise}
\label{xca:invertible_algebraic}
	Let $A$ be an algebraic algebra and $a\in A$.
	\begin{enumerate}
		\item $a$ is a left zero divisor if and only if $a$ is a right zero divisor.
		\item $a$ is left invertible if and only if $a$ is right invertible.
		\item $a$ is invertible if and only if $a$ is not a zero divisor.
	\end{enumerate}
\end{exercise}

\begin{exercise}
	\label{exa:norma}
	For $\alpha=\sum_{g\in G}\alpha_gg\in\C[G]$ let $|\alpha|=\sum_{g\in
	G}|\alpha_g|\in\R$. Prove the following statements:
	\begin{enumerate}
		%\item $|\trace(\alpha)|\leq |\alpha|$, 
		\item $|\alpha+\beta|\leq|\alpha|+|\beta|$, and 
		\item $|\alpha\beta|\leq|\alpha||\beta|$ 
	\end{enumerate}
	for all $\alpha,\beta\in\C[G]$.
\end{exercise}

\begin{theorem}[Formanek]
	\label{thm:FormanekQ}
	\index{Formanek's theorem}
	Let $G$ be a group. If every element of $\Q[G]$ is invertible or 
	a zero divisor, then $G$ is locally finite. 
\end{theorem}

\begin{proof}
	Let $\{x_1,\dots,x_n\}$ be a finite subset of $G$. Adding inverses if needed, we may assume that 
	$\{x_1,\dots,x_n\}$ generates the subgroup
	$H=\langle x_1,\dots,x_n\rangle$ as a semigroup. Let 
	\[
		\alpha=\frac{1}{2n}(x_1+\cdots+x_n)\in\Q[G]
	\]
    Note that $|\alpha|\leq 1/2$. 
	We claim that $1-\alpha\in\Q[G]$ is invertible. If not, then it is a zero divisor. If there exists 
	$\delta\in\Q[G]$ such that $\delta(1-\alpha)=0$, then 
	$\delta=\delta\alpha$. Since  
	\[
		|\delta|=|\delta\alpha|\leq|\delta||\alpha|\leq|\delta|/2,
	\]
	it follows that $\delta=0$. Similarly, $(1-\alpha)\delta=0$ implies
	$\delta=0$. 
	
	Let $\beta=(1-\alpha)^{-1}\in\Q[G]$.  For each $k$ let  
	\[
		\gamma_k=(1+\alpha+\cdots+\alpha^k)-\beta.
	\]
	Then 
	\begin{align*}
		\gamma_k(1-\alpha)&=(1+\alpha+\cdots+\alpha^k-\beta)(1-\alpha)\\
		&=(1+\alpha+\cdots+\alpha^k)(1-\alpha)-\beta(1-\alpha)=-\alpha^{k+1}
	\end{align*}
	and thus  
	$\gamma_k=-\alpha^{k+1}\beta$. Since  
	\[
		|\gamma_k|=|-\alpha^{k+1}\beta|\leq|\beta||\alpha^{k+1}|\leq\frac{|\beta|}{2^{k+1}},
	\]
	it follows that $\lim_{k\to\infty}|\gamma_k|=0$. 

	We now prove that $H\subseteq\supp\beta$. This will finish the proof of the theorem, 
	as $\supp\beta$ is a finite subset of $G$ by definition. If
	$H\not\subseteq\supp\beta$, let $h\in H\setminus\supp\beta$.  Assume that 
    $h=x_{i_1}\cdots x_{i_m}$ is a word in the letters $x_j$ of length $m$. Let 
    $c_j$ be the coefficient of $h$ in $\alpha^j$. Then $c_0+\cdots+c_k$ is the 
	coefficient of $h$ in $\gamma_k$, but 
	\[
		|\gamma_k|\geq c_0+c_1+\cdots+c_k\geq c_m>0
	\]
	for all $k\geq m$, as each $c_j$ is non-negative, a contradiction to 
	$|\gamma_k|\to 0$ si $k\to\infty$.
\end{proof}

